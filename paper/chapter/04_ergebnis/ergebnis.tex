\chapter{Ergebnis}\label{ch:ergebnis}
% \begin{itemize}
%   \item Alle Plots zeigen (oder nur eine Teilmenge?)
%     \begin{itemize}
%       \item Bei den Plots schwache Punkte gar nichts erst anzeigen?!
%       \item Auf jedenfall gute immer beschreiben (welche Parameter z.B.)
%     \end{itemize}
%   \item Auflisten welche Action-Unit welche Classificator gut war (Recall, Precision, F1-Score)
%   \item Klar machen, wie gut diese Klassifikatoren bei Trainingsmenge abschneiden
%   \item Allgemeine Aussage, welche Klassifikatoren mit Parametern überhaupt nicht geeignet sind und welche super sind.
%   \item Aussage welche Action-Unit gut zu klassifizieren ist
%   \item Erwähnen, dass Shuffle SVM-Ergebnisse ändert.
% \end{itemize}

Bei den hier folgenden Ergebnissen, wurden die Videos durch Rauschen künstlich erweitert und normalsiert. Danach wurden die Features aus den resultierenden Landmarks extrahiert. Es folte eine Normalisierung der Features anhand ihres Mittelwerts und ihrer Varianz. Für eine Reduktion des Fetaure-Raumes auf eine verbleibenden Varianz von $97\%$ wurde dann der PCA-Algorithmus angewendet. Eine Ausnahme ist hierbei die Interpolation, da dieses Feature schon den Raum verkleinert hat, wurde auf eine weitere Reduktion durch PCA verzichtet. Schlussendlich wurden die Features dann zufällig geshuffled. Für zeitlichen Features haben wir uns nur aufX-/Y-Koordinaten und die Interpolationskoeffizienten konzentriert. Für die Ableitung wurde $\delta=2$ gesetzt.
Alle diese Methoden wurden in Kapitel \ref{ch:methodik} vorgestellt.

Wir haben dann anhand dieser Feature mehrere SVM mit den verschiedenen vorgestellten Kernelfunktionen trainiert. Zusätzlich haben wir auch verschiedene RF getestet, indem wir die Anzahl der Bäume und ihre maximale Tiefe varrierten.

\begin{table}[h]
  \centering
  \pgfplotstabletypeset[
  col sep=comma,
  every head row/.style={
    before row={
      \toprule
      \multirow{2}{*}{AU} & \multicolumn{6}{c}{Bester Klassifikator} \\
      % & F1 Val & F1 Test & Precision Test & Recall Test & Features \\
    },
    % output empty row,
    after row=\midrule,
  },
  precision=3,
  fixed,
  empty cells with={--},
  sort, sort key=BestF1Val, sort cmp=float >,
  columns={AU, BestF1Val, F1Test, PrecisionTest, RecallTest, BestClassifier, BestFeatures},
  columns/AU/.style={string type, column name={}, column type/.add={}{|}},
  columns/BestF1Val/.style={column name={F1 Val}},
  columns/F1Test/.style={clear infinite, column name={F1 Test}},
  columns/PrecisionTest/.style={column name={Prec. Test}},
  columns/RecallTest/.style={column name={Recall Test}},
  columns/BestClassifier/.style={string type, column name={Klassifikator}},
  columns/BestParams/.style={string type, column name={Parameter}},
  columns/BestFeatures/.style={string type, column name={Features}}
  ]{data/all_au_top1.dat}
  
  \caption{F1 scores und Testergebnisse des besten Klassifikators pro Action Unit}
  \label{tbl:all_au_res}
\end{table}

In \cref{tbl:all_au_res} sind die Ergebnisse des besten Klassifikators
(ausgewählt nach F1 score auf der Validierungsmenge) pro Action Unit zu sehen.
Gute Klassifikation auf unbekannten Personen ist möglich für Lips Part: Der hohe
F1 score $0.656$ in der Validierung bestätigt sich auch auf der Testmenge. Akzeptable
Performance liefert der beste Klassifikator für Lip Corner Puller. Dieser
verbessert sich sogar von einem F1 score von $0.37$ in der Validierung auf
$0.431$ im Test.

Die Klassifikatoren für Outer Brow Raiser und Upper Lid Raiser scheinen in der
Validierung akzeptabel, erkennen jedoch im Test überhaupt keine Aktivierung der
Action Units mehr. Die übrigen Action Units werden mit keiner unserer Feature
Extraction-Methoden an neuen Personen befriedigend klassifiziert.
\begin{figure}
  \centering
  \begin{tikzpicture}
    \begin{axis}[
      title=Lips Part,
      width=0.5\textwidth,
      only marks,
      mark size=1.5pt,
      xlabel=$1-\text{Precision}$,
      ylabel=Recall,
      legend entries={SVM, Random Forest},
      legend pos=north west
      ]
     \addplot+ [x filter/.expression={\thisrow{IsSVM}==1 ? x : nan}] table [x index=1, y index=2,x expr={1-\thisrow{Precision}}] {data/validation/Lips_Part.dat};
     \addplot+ [x filter/.expression={\thisrow{IsSVM}==0 ? x : nan}] table [x index=1, y index=2,x expr={1-\thisrow{Precision}}] {data/validation/Lips_Part.dat};
    \end{axis}
  \end{tikzpicture}
  ~
  \begin{tikzpicture}
    \begin{axis}[
      title=Lip Corner Puller,
      width=0.5\textwidth,
      only marks,
      mark size=1.5pt,
      xlabel=$1-\text{Precision}$,
      ylabel=Recall,
      legend entries={SVM, Random Forest},
      legend pos=north west
      ]
     \addplot+ [x filter/.expression={\thisrow{IsSVM}==1 ? x : nan}] table [x index=1, y index=2,x expr={1-\thisrow{Precision}}] {data/validation/Lip_Corner_Puller.dat};
     \addplot+ [x filter/.expression={\thisrow{IsSVM}==0 ? x : nan}] table [x index=1, y index=2,x expr={1-\thisrow{Precision}}] {data/validation/Lip_Corner_Puller.dat};
    \end{axis}
  \end{tikzpicture}
  \caption{Validierungsergebnisse für Lips Part und Lip Corner Puller. Jeder Punkt steht für eine Kombination aus Feature Extraction, Klassifikator und Parametern.}
\end{figure}




\cref{fig:lips_prec_recall} zeigt, dass die verschiedenen Kombinationen aus
Feature Extraction, Klassifikator und Parametern zu stark variierender
Performance auf der Validierungsmenge führen. Der Tradeoff zwischen Precision
und Recall ist deutlich zu sehen. Man sieht eine Trennung zwischen SVM und
Random Forest Klassifikatoren: erstere tendieren dazu, zu viele negative Frames
(ohne Aktivierung der Action Unit) als positiv zu klassifizieren, was zu
schlechter Precision führt. Die Random Forests neigen hingegen zu vielen False
Negatives. Sie schneiden bezogen auf Lips Part besser ab, bei anderen Action
Units ist die Performance zwischen den Klassifikatoren ausgeglichen.
Aus Platzgründen werden hier nur Graphen für ausgewählte Action Units gezeigt.
Unsere gesamten Ergebnisse sind aber auf einer von uns
erstellten Internetseite veröffentlicht
\footurlcite{web}.

\begin{table}[h]
\centering
\pgfplotstabletypeset[
  precision=3,
  col sep=comma,
  fixed zerofill,
  columns={F1Validation, F1Test, PrecisionTest, RecallTest, Classifier, Params, Features},
  columns/F1Validation/.style={column name={F1 Val}},
  columns/F1Test/.style={column name={F1 Test}},
  columns/PrecisionTest/.style={column name={Prec. Test}},
  columns/RecallTest/.style={column name={Recall Test}},
  columns/Classifier/.style={string type, column name={Klassif.}},
  columns/Params/.style={string type, column name={Parameter}},
  columns/Features/.style={string type, column name={Features}}
  ]{data/top5/Lips_Part.dat}
  \caption{F1 scores und Testergebnisse der Top 5 Klassifikatoren für Lips Part}
  \label{tbl:lips_part_top}
\end{table}

\begin{table}[h]
\centering
\pgfplotstabletypeset[
  precision=3,
  col sep=comma,
  fixed zerofill,
  columns={F1Validation, F1Test, PrecisionTest, RecallTest, Classifier, Params, Features},
  columns/F1Validation/.style={column name={F1 Val}},
  columns/F1Test/.style={column name={F1 Test}},
  columns/PrecisionTest/.style={column name={Prec. Test}},
  columns/RecallTest/.style={column name={Recall Test}},
  columns/Classifier/.style={string type, column name={Klassif.}},
  columns/Params/.style={string type, column name={Parameter}},
  columns/Features/.style={string type, column name={Features}}
  ]{data/top5/Lip_Corner_Puller.dat}
  \caption{F1 scores und Testergebnisse der Top 5 Klassifikatoren für Lip Corner Puller}
  \label{tbl:lip_corner_top}
\end{table}
Die Dominanz der Random Forests für Lips Part ist auch in
\cref{tbl:lip_corner_top} zu sehen. Die beste SVM taucht mit einem F1 score von
$0.356$ in den Top 5 nicht mehr auf. Lip Corner Puller wird dagegen von einer
SVM am besten erkannt, allerdings dominieren wieder Random Forests in den Top 5.

\begin{table}[h]
  \centering
  \pgfplotstabletypeset[
  col sep=comma,
  % every head row/.style={
  %   before row={
  %     \toprule
  %     \multirow{2}{*}{AU} & \multicolumn{5}{c}{Bester Klassifikator} \\
  %     & F1 Val & F1 Test & Precision Test & Recall Test & Features \\
  %   },
  %   output empty row,
  %   after row=\midrule,
  % },
  precision=3,
  fixed,
  empty cells with={--},
  sort, sort key=BestF1, sort cmp=float >,
  columns={Features, BestF1, Dims, DimsAfterPCA},
  columns/Features/.style={string type, column name={Features},
    postproc cell content/.code={%
          \pgfplotsutilstrreplace{_}{\_}{##1}%
          \pgfkeyslet{/pgfplots/table/@cell content}\pgfplotsretval
      },},
  columns/Dims/.style={column name={Dimensionen}},
  columns/DimsAfterPCA/.style={clear infinite, column name={Dim. nach PCA}},
  columns/BestF1/.style={column name={Bester F1 score}}
  ]{data/features/Lips_Part.dat}
  
  \caption{Vergleich der verschiedenen Features für Lips Part. Auf Interpolation-Features wurde keine PCA angewandt.}
  \label{tbl:features}
\end{table}
Schließlich liefert \cref{tbl:features} einen Vergleich der verschiedenen
Features am Beispiel von Lips Part. Der Vergleichswert ist ein F1 score von
$0.585$, wenn man die Koordinaten aller Landmarks als Features übernimmt (``XY''
in der Tabelle). Besser schneiden die Features ``CenterDistance'' und
``Interpolation'' ab. Es fällt auf, dass die PCA die Dimensionen von
``Orientation'' überhaupt nicht und von ``CenterOrientation'' fast vollständig reduziert. Die zeitbasierten Features bilden das Schlusslicht in den
F1-Wertungen, mehr dazu in der Diskussion.

%%% Local Variables: 
%%% mode: latex
%%% TeX-master: "../../paper"
%%% End: 
