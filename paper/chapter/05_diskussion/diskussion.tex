\chapter{Diskussion}
% \begin{itemize}
%   \item Wieso sind gut bei Test, aber schlecht bei Training 
%   \item Wieso sind diese Klassifikatoren gut und andere nicht
%    \item Wieso sind viele Action-Units schlecht zu erkennen
%    \item Wieso gerade diese Feature so gut?
%   \item $\Rightarrow$ Overfitting
%   \item Warum geht diese Action-Unit besser als andere
% \end{itemize}

In den Ergebnissen haben wir gesehen, dass Lips Part und Lip Corner Puller
relativ gut, die anderen Action Units eher schlecht erkannt werden. Lips Part
ist leich zu erkennen, da nur die Abstände der 6 Landmarks in der Mitte des
Mundes relevant sind:
\begin{figure}[h]
  \centering
\begin{tikzpicture}
  \begin{axis}[
    width=7cm,
    height=4cm,
    ticks=none,
    y dir=reverse,
    axis lines=none
    ]
    \addplot [
    draw=red,
    mark=*,
    only marks,
    mark size=1pt,
    nodes near coords={
      \pgfmathparse{int(\coordindex+48)}
      \pgfmathresult
    },
    every node near coord/.style={
        font=\scriptsize,
        anchor=south west,
        xshift=-2pt,
        yshift=-3pt
      },
      select coords between index={60}{65},
    ] table {data/face_lipcorner_low.dat};
    \addplot [
    % fill=blue,
    mark=o,
    only marks,
    mark size=1pt,
    select coords between index={48}{59},
    ] table {data/face_lipcorner_low.dat};
  \end{axis}
  \end{tikzpicture}
  \caption{Einfach Klassifikation von Lips Part}
\end{figure}

Dies erklärt auch, warum die eher simple CenterDistanceExtraction erfolgreich
ist: Ein Vergleich der Abstände von Landmarks 49 und 52 zum Mittelpunkt liefert
ein einfaches Kriterium, ob die Action Unit aktiviert ist.
In den Top 5 zu Lip Corner Puller taucht häufig die InterpolationExtraction auf.
Wie in \cref{fig:interpol} dargestellt, kann durch Polynominterpolation gut die
Form und Öffnungsrichtung von Ober- und Unterlippe extrahiert werden. 

Eine weitere Erklärung in Bezug auf die Performanceunterschiede für verschiedene
Action Units liefern \cref{fig:f1vspos,tbl:positives}. Die Anzahl Frames, in
denen die Action Unit aktiviert ist, variiert stark zwischen den Action Units.
Nicht überraschend sind Lips Part und Lip Corner Puller in vergleichsweise
vielen Frames aktiv. Upper Lid Raiser und Outer Brow Raiser lieferten
zwar gute F1 scores in der Validierung, aber auf der Testmenge war die
Performance schlecht (vlg. \cref{tbl:all_au_res}). Auch dies stimmt mit einer
geringen Anzahl positiver Frames überein. Möglicherweise wurde durch die Auswahl
des besten Klassifikators aus 160 die Validierungsmenge overfitted.

Die Action Unit Brow Lowerer fällt hier aus der Reihe, da sie sehr schlecht
klassifiziert wurde, obwohl viele positive Frames vorhanden waren.
Wahrscheinlich haben wir hierfür nicht die richtigen Features gefunden.
\begin{figure}
  \centering
  \begin{tikzpicture}
    \begin{axis}[
      % title=,
      only marks,
      mark size=1.5pt,
      % xlabel=$\text{\#Positives} / \text{\#Samples}$,
      xlabel=\#Positives,
      ylabel=F1 score,
      legend entries={Validation, Test},
      legend pos=north west
      ]
     \addplot+ table [col sep=comma, x=PosRatio, y=BestF1Val] {data/all_au_top1.dat};
     \addplot+ table [col sep=comma, x=PosRatio, y=F1Test] {data/all_au_top1.dat};
    \end{axis}
  \end{tikzpicture}
  \caption{F1 score des besten Klassifikators jeder Action Unit gegen Anteil positiver Samples}
\end{figure}

\begin{table}[h]
  \centering
  \pgfplotstabletypeset[
  col sep=comma,
  every head row/.style={
    before row={
      \toprule
      \multirow{2}{*}{AU} & \multirow{2}{*}{F1 Val} & \multicolumn{2}{c}{Positives in Training} \\
    },
    % output empty row,
    after row=\midrule,
  },
  precision=3,
  fixed,
  empty cells with={--},
  sort, sort key=BestF1Val, sort cmp=float >,
  columns={AU, BestF1Val, NumPositives, PosRatio},
  columns/AU/.style={string type, column name={}, column type/.add={}{|}},
  columns/BestF1Val/.style={column name={}},
  columns/NumPositives/.style={column name={\#}},
  columns/PosRatio/.style={column name={\%},preproc/expr={100*##1},precision=1},
  ]{data/all_au_top1.dat}
  
  \caption{Gegenüberstellung der F1 scores und der Häufigkeit positiver Trainingsbeispiele}
  \label{tbl:positives}
\end{table}

Die zeitbasierte Klassifikation war wenig erfolgreich. Auch hier ist ein Blick
auf das Verhältnis positiver Frames aufschlussreich:
\begin{table}[H]
  \centering
  \pgfplotstabletypeset[
  col sep=comma,
  every head row/.style={
    before row={
      \toprule
      \multirow{2}{*}{AU} & \multicolumn{2}{c}{Positives in Training} \\
    },
    % output empty row,
    after row=\midrule,
  },
  precision=3,
  fixed,
  empty cells with={--},
  sort, sort key=NumPositives, sort cmp=float >,
  columns={AU, NumPositives, PosRatio},
  columns/AU/.style={string type, column name={}, column type/.add={}{|}},
  columns/NumPositives/.style={column name={\#}},
  columns/PosRatio/.style={column name={\%},preproc/expr={100*##1},precision=1},
  ]{data/time_positives.dat}
  
  \caption{Positive Trainingsbeispiele für zeitbasierte Features}
  \label{tbl:time_positives}
\end{table}
In den Ergebnissen ist offensichtlich, dass schon die Klassifikation der
\emph{aktuellen} Action Unit im Frame schwer ist. Deshalb ist es nicht verwunderlich,
dass die Vorhersage, ob sich die Action Unit gerade ändert, noch schwerer zu
treffen ist.

Insgesamt ist festzuhalten, dass die Generalisierung auf Personen, die nicht im
Training gesehen wurden, eine schwierige Aufgabe ist. Wie \cref{tbl:trainvstest}
zeigt, wurden während des Testlaufs für jede Action Unit mehrere Klassifikatoren
gefunden, die auf dem Trainingsdatensatz fast perfekte Performance erreichen.
Allerdings hat sich diese Performance nur in den wenigen bereits diskutierten Fällen
auf die neuen Personen in Validierungs- und Testmenge übertragen. Die
Klassifikatoren scheinen sich also die individuellen Formen und Abstände in den Gesichtern
der Trainingspersonen gemerkt zu haben (Overfitting).

%%% Local Variables: 
%%% mode: latex
%%% TeX-master: "../../paper"
%%% End: 

