\chapter{Fazit}\label{ch:fazit}
\section{Zusammenfassung}
Die Aufgabe des Praktikums war es, Action Units in Gesichtern von Personen mittels der vorgegebenen
Landmarks zu klassifizieren.
Wir haben uns zusätzlich die Frage gestellt: Wie gut lässt sich ein Klassifikator
trainieren, der Action Units bei unbekannten Personen erkennen kann?

Dazu haben wir uns sieben verschiedene Methoden überlegt, um aus den Landmarks
aussagekräftige Features zu extrahieren. Diese reichen von eher simplen Features (Distanz
zum Mittelpunkt), zu komplexeren wie die Polynominterpolation und die zeitliche
Ableitung. Wir haben eine Pipeline gebaut, die es ermöglicht, viele
Kombinationen aus Features, Klassifikationsalgorithmen und Parametern auf allen
Action Units zu testen.
Zur Evaluierung haben wir sinnvolle Performancestatistiken ausgesucht, die
Klassifikatoren verglichen und die Ergebnisse visualisiert und diskutiert.

Wir stellen fest, dass die Verallgemeinerung auf im Training nicht gesehene
Personen nur bei zwei Action Units befriedigend funktioniert hat. Dies liegt an
der ungenügenden Anzahl verschiedener Personen in der Trainingsmenge sowie der
geringen Anzahl von Frames, in denen manche Action Units aktiviert waren.
Erfolgreich war die Erkennung von Lips Part und Lip Corner Puller. Hier haben
auch die von uns entwickelten Features (Distanz zum Mittelpunkt bzw.
Interpolation) gute Ergebnisse erziehlt.
\section{Ausblick}
% \begin{itemize}
%   \item Was könnte man noch verbesser, und wieso haben wir das nicht gemacht (z.B. aus Zeitgründe)
%     \begin{enumerate}
%       \item Mehr Kombinationen (Mit/ohne PCA, mehr Time-Differential-Feature, überhaupt mehr zeitliche Features, andere normalisierungen der Punktwolke, Neuronales-Netzwerk oder andere Klassifikatoren dazu benutzen)
%     \end{enumerate}
%   \item Wie könnte das Ergebnis besser werden (z.B. mehr Daten von mehreren Personen)
% \end{itemize}
Wir wollen uns hier die Frage stellen, wie unser Ergebnis weiter verbessert werden könnte.

Um auch bei anderen Action Units bessere Ergebnisse zu erzielen, muss für
größere Varianz in der Trainingsmenge gesorgt werden. Im Praktikum standen nur
15 Videos (davon nur 10 in der ersten Phase) zur Verfügung, was dazu geführt
hat, dass in der Trainingsmenge nur sechs Personen vertreten waren. Die volle
DISFA Datenbank bietet Videos von 27 Personen. Hier gibt es also mit mehr Daten
noch Verbesserungspotential.

Im Praktikum wurden nur die Landmarks als Rohdaten benutzt. Allerdings enthalten
die vollen Videos mehr Informationen, die sich mit Methoden der Computer Vision
(Gabor Filter etc.) extrahieren ließen. Diese sind teilweise schon mit der DISFA mitgeliefert.

Die Klassifikation anhand von weiteren zeitbasierten Features könnte noch ausgebaut
werden. Dies ist aus Zeitgründen nicht geschehen. Allerdings müsste dazu die
Einordnung eines Frames als aktiv bzw. nicht aktiv überdacht werden, damit es
mehr positive Frames gibt.

Zusätzlich gibt es noch die Möglichkeit mehrere Features und Klassifikatoren zu kombinieren um verlässlichere Ergebnisse zu erzielen.

%%% Local Variables: 
%%% mode: latex
%%% TeX-master: "../../paper"
%%% End: 
