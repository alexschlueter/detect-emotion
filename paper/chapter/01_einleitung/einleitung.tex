\chapter{Einleitung}

Das Thema der (visuellen) Emotionserkennung durch Computersysteme hat in den letzten Jahren immer mehr an Bedeutung gewonnen.
Die Einsatzgebiete sind vielseitig und reichen von Sicherheitsanwendungen, über Robotik, bis hin zu Unterhaltungsmedien.
Meist wird versucht, anhand von verschiedenen Merkmalen im Gesicht, diesem eine oder mehrere Emotionen zuzuordnen. Im Rahmen
unsere Praktikums, war es unsere Aufgabe ein solches Computersystem zur Erkennung von Emotionen zu entwickeln.\newline
Im Folgenden Kapitel werden die Aufgabenstellung, sowie die Eingabedaten genauer beschrieben.
In Kapitel 2 werden daraufhin die Methodiken vorgestellt, die wir für unser Programm nutzen, woraufhin in Kapitel 3 erläutert wird,
wie wir diese implementiert haben. Daraufhin folgt die Vorstellung unserer Ergebnisse in Kapitel 4 und ein Ausblick auf mögliche
Erweiterungen der Anwendung. Im letzten und 5. Kapitel wird das Ergebnis der Arbeit kurz resümiert.

\section{Facial Action Code}
Der Facial Action Code (kurz FAC) ist ein System zur Unterscheidung von Bewegungen von isolierten Teilen des menschlichen Gesichts, welches
1976 von Paul Ekman und Wallace V. Friesen entwickelt wurde. Es basiert auf sogenannten Action Units (kurz AU), welche eben genau diese Bewegungen
beschreiben sollen. Dabei kann eine Action Unit eine Ausprägung zwischen einschließlich 0 und 5 haben, wobei 0 bedeutet, dass keine entsprechende
Bewegung vorhanden ist, und 5 bedeutet, dass die Bewegung maximal stark ausgeprägt ist \cite{ekman}. Eine Auflistung der für diese Arbeit relevanten Action Units
findet sich im Anhang \ref{Anhang.AUs}.

\section{DISFA Datenbank}
Die Denver Intensity of Spontaneous Facial Action Database (kurz DISFA Database) enthält eine Sammlung von Gesichtsbewegungen von insgesamt
27 unterschiedlichen, erwachsenen Probanden. Hierzu wurde von jedem Probanden ein 4-minütiges Video mit je 20 Frames pro Sekunde gedreht.
Danach wurde jedes Frame nach dem Facial Action Coding System auf die Ausprägung von 12 Action Units analysiert und gelabelled.\newline
Weiterhin enthält jedes Frame 66 Landmark Koordinaten, von markanten Punkten des Gesichtes.

\section{Die Aufgabenstellung}
Die Aufgabenstellung des Praktikums bestand darin, aus einer Auswahl von 12 Videos der DISFA Datenbank einen Klassifikator zu entwickeln,
der möglichst präzise in der Lage ist für ein beliebiges Frame aus der Datenbank zu bestimmen, welche Action Units in dem Frame aktiviert
sind (dh. eine Ausprägung größer oder gleich 1 haben).
