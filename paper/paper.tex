%!TEX TS-program = pdflatex
%!BIB program = biber
% -- Author: Jannes Bantje, j.bantje@wwu.de
%!TEX root = bachelor.tex
% -- Author: Jannes Bantje, j.bantje@wwu.de
\documentclass[a4paper,ngerman,index=totoc,toc=bibliography,fontsize=10,DIV=13,headinclude,twoside,BCOR=12mm,cleardoublepage=empty,draft]{scrreprt}


%-- Basics für graphische Sachen
\usepackage[usenames,x11names]{xcolor} % Die Optionen definieren zusätzliche Farben (siehe Dokumentation)
\usepackage[final]{graphicx}

%-- typografische Verbesserungen, Codierungskram, Schriftwahl und erste Mathepakete
\usepackage[utf8]{inputenc}
\usepackage[lining,semibold]{libertine}
\usepackage[T1]{fontenc}
\usepackage{textcomp} % verhindert ein paar Fehler bei den Fonts
\usepackage[varl]{zi4}
\usepackage{mathtools,amssymb,amsthm} % Verbesserung von amsmath (die amsmath selbst lädt)
\usepackage[libertine,cmintegrals,bigdelims,varbb]{newtxmath}
\usepackage[ngerman]{babel}
\usepackage[babel=true, tracking=true,final]{microtype}

% \useosf % aktiviert sog. "old style figures", also werden Zahlen – im Text – teilweise unterhalb der Grundlinie angezeigt. Muss man mögen...


%-- Alternative Konfiguration mit XeLaTeX, die ein großes "ß" anzeigen kann!
%-- ACHTUNG: wenn diese benutzt werden soll, dan MUSS man in der ersten Zeile von bachelor.tex "pdflatex" durch "xelatex" ersetzen und natürlich den obigen Block auskommentieren und diesen wiederrum aktivieren!!!

% \usepackage{lmodern}
% \usepackage{mathtools,amssymb,amsthm} % Verbesserung von amsmath (die amsmath selbst lädt)
% \usepackage[libertine,cmintegrals,bigdelims,varbb]{newtxmath}
% \usepackage[no-math]{fontspec}
% \usepackage{polyglossia} % moderner babel-ersatz
% \setmainlanguage[spelling=new,babelshorthands=true]{german}
% \shorthandoff{"}
% \defaultfontfeatures{Mapping=tex-text, Ligatures={Required,Common,Contextual}}
% \setmainfont{LinLibertine}[Extension=.otf,UprightFont=*_R,BoldFont=*_RZ,ItalicFont=*_RI,BoldItalicFont=*_RZI,ItalicFeatures={Ligatures=Historical}]
% \setsansfont{LinBiolinum}[Scale=MatchUppercase, Extension=.otf, UprightFont=*_R, BoldFont=*_RB, ItalicFont=*_RI,BoldItalicFont=*_RBO]
% \setmonofont{Inconsolatazi4}[Scale=MatchUppercase,Extension=.otf,UprightFont=*-Regular,BoldFont=*-Bold,StylisticSet=1]
% \usepackage[final]{microtype}


%-- Zeilenabstand einstellen
\usepackage{setspace}
% Nun kann man, wenn gewünscht den Zeilenabstand zum Beispiel auf 1,5 setzen mit \onehalfspacing

\newcommand{\command}[1]{\texttt{\textbackslash{}#1}}



%-- Mathematikpakete und Einstellungen
\mathtoolsset{centercolon} % sorgt dafür dass := und =: besser aussehen
\usepackage{mathdots} % sorgt dafür, dass Punte wie zB \ddots besser aussehen
\newcommand{\Underbrace}[2]{{\underbrace{#1}_{#2}}} % Underbrace als Befehl in LaTeX-Syntax (und ohne Spacing-Probleme mit nachfolgenden Operatoren...)
\renewcommand{\le}{\leqslant} % ich finde Kleinergleich mit schrägen Strich schöner
\renewcommand{\ge}{\geqslant}

%-- charakteristische-Funktion-/Indikatorfunktion-Eins '\ind'
\usepackage{silence}
\WarningFilter{latexfont}{Size substitutions with differences}
\WarningFilter{latexfont}{Font shape `U/bbold/m/n' in size}
\DeclareSymbolFont{bbold}{U}{bbold}{m}{n}
\DeclareSymbolFontAlphabet{\mathbbold}{bbold}
\newcommand{\ind}{\mathbbold{1}}

%-- Ein sehr hübscher Mengen-Befehl
\newcommand\SetSymbol[1][]{\nonscript\:#1\vert\allowbreak\nonscript\:\mathopen{}}
\providecommand\given{} % to make it exist
\DeclarePairedDelimiterX\set[1]\{\}{\renewcommand\given{\SetSymbol[\delimsize]}#1}

%-- Klammern, Skalarprodukt und Norm
\DeclarePairedDelimiter{\enbrace}{(}{)}
\DeclarePairedDelimiter{\abs}{|}{|}
\DeclarePairedDelimiterX\skal[2]{\langle}{\rangle}{#1\,\delimsize\vert\,#2}
\DeclarePairedDelimiter{\norm}{\lVert}{\rVert}

%-- Differentialrechnung
\newcommand{\mathd}{\mathrm{d}\mkern-0.5mu}
\newcommand{\diff}[2]{\frac{{\partial #1}}{{\partial #2}} }
\newcommand{\diffd}[2]{\frac{\mathd #1}{\mathd #2} }

%-- eigene Befehle
\DeclareMathOperator{\sgn}{sgn}


%-- kommutative Diagramme
\usepackage{tikz-cd} %-- meiner Meinung nach das beste Paket für kommutative Diagramme
\tikzset{% um Kompatibilität mit Babel herzustellen und die angenehme "<label>"-Syntax zu nutzen
  every picture/.append style={
    execute at begin picture={\shorthandoff{"}},
    execute at end picture={\shorthandon{"}}
  }
}
\usetikzlibrary{quotes,babel}

%-- Für Literaturangaben, hier wird NICHT das total veraltete bibtex benutzt!
\usepackage[%
	backend=biber,
	sortlocale=auto,
	natbib,
	hyperref,
	% backref,
	style=alphabetic % eine unvollständige Auswahl von Styles: ieee, numeric, apa
	]%
{biblatex}
\addbibresource{quellen.bib} % Literaturdatei einlesen

% -- Konfiguration von Hyperref (sorgt für anklickbare Links und ein PDF-Inhaltsverzeichnis)
\usepackage[hidelinks, pdfpagelabels, bookmarksopen=true, bookmarksnumbered=true, linkcolor=black, urlcolor=SkyBlue2, plainpages=false,pagebackref, citecolor=black, hypertexnames=true, pdfborderstyle={/S/U}, linkbordercolor=SkyBlue2, colorlinks=false, backref=false]{hyperref}
\hypersetup{final}

%-- Für Aufzählungen und andere Listen, Anführungszeichen und Zitate
\usepackage[shortlabels]{enumitem} % durch die Option ist die gleiche Syntax wie zB mit dem Paket paralist möglich
\setlist[enumerate,description]{font=\sffamily\bfseries} % sorgt dafür, dass die Labels bei enumerate und description fett sind
\usepackage[german=quotes]{csquotes}

%-- Für hilfreiche Anmerkungen am Seitenrand
\usepackage[obeyDraft,textsize=small]{todonotes}

%-- Kopf- und Fußzeilen bearbeiten
\usepackage{scrpage2}
\pagestyle{scrheadings}
\clearscrheadfoot % Standardkonfiguration löschen
\setheadsepline{1pt} % Linie für die Kopfzeile
\automark[section]{chapter} % definiert, welcher Text in den Kolumnentiteln erscheinen soll
\rohead{\rightmark} % section erscheint rechts oben
\lehead{\scshape\leftmark} % chapter erscheint links oben in ist in small caps gesetzt
\ofoot[\pagemark]{\pagemark} % Seitenzahlen immer außen, hier wir auch der plain Stil bearbeitet!
% \ifoot[Titel der Bachelorarbeit]{Titel der Bachelorarbeit}
\renewcommand*{\pnumfont}{\LARGE\sffamily} % Seitenzahlen in groß und serifenlos
\renewcommand*{\footfont}{\large\sffamily\color{gray}}
% \renewcommand*{\headfont}{\normalfont}



%-- Theorem-Pakete und Konfiguration
\usepackage{thmtools}

\usepackage{bookmark}
% Theoreme als PDF-Lesezeichen
\bookmarksetup{open,numbered}
\makeatletter
\newcommand*{\theorembookmark}{%
  \bookmark[
    dest=\@currentHref,
    rellevel=1,
    keeplevel,
  ]{%
    \thmt@thmname\space\csname the\thmt@envname\endcsname
    \ifx\thmt@shortoptarg\@empty
    \else
      \space(\thmt@shortoptarg)%
    \fi
  }%
}
\makeatother



\declaretheoremstyle[%
	headfont=\sffamily\bfseries,
	notefont=\normalfont\sffamily,
	bodyfont=\normalfont,
	headformat=\NUMBER\ \NAME\NOTE,
	headpunct={},
	postheadspace=1ex,
	postheadhook=\theorembookmark,
	spaceabove=15pt,spacebelow=10pt,]%
{mainstyle}
\declaretheoremstyle[%
	headfont=\bfseries\scshape,
	bodyfont=\normalfont,
	headpunct=:,
	postheadspace=1ex,
	spacebelow=12pt,spaceabove=2pt,
	qed=\qedsymbol]%
{beweise}

\declaretheorem[name=Definition,parent=section,style=mainstyle]{definition}
\declaretheorem[name=Satz,sharenumber=definition,style=mainstyle]{satz}
\declaretheorem[name=Korollar,sharenumber=definition,style=mainstyle]{korollar}
\declaretheorem[name=Lemma,sharenumber=definition,style=mainstyle]{lemma}
\declaretheorem[name=Proposition,sharenumber=definition,style=mainstyle]{proposition}

\declaretheorem[name=Beweis,numbered=no,style=beweise]{beweis}

\usepackage{titling}
\usepackage{pgfplots}
\usepackage{pgfplotstable}

\pgfplotsset{compat=1.13}
\usetikzlibrary{arrows,decorations.markings,positioning,calc,}

\pgfplotsset{
% Style to select only points from #1 to #2 (inclusive)
  select coords between index/.style 2 args={
    x filter/.code={
        \ifnum\coordindex<#1\def\pgfmathresult{}\fi
        \ifnum\coordindex>#2\def\pgfmathresult{}\fi
    }
},
  discard if not/.style 2 args={
      x filter/.code={
          \edef\tempa{\thisrow{#1}}
          \edef\tempb{#2}
          \ifx\tempa\tempb
          \else
              \def\pgfmathresult{inf}
          \fi
      }
    },
    filter discard warning=false
}

\newcount\colveccount
\newcommand*\colvec[1]{
        \global\colveccount#1
        \begin{pmatrix}
        \colvecnext
}
\def\colvecnext#1{
        #1
        \global\advance\colveccount-1
        \ifnum\colveccount>0
                \\
                \expandafter\colvecnext
        \else
                \end{pmatrix}
        \fi
}
\newcommand{\myarrowopts}{[thick, decoration={markings,mark=at position
   1 with {\arrow[semithick]{open triangle 60}}},
   double distance=1.4pt, shorten >= 5.5pt,
   preaction = {decorate},
   postaction = {draw,line width=1.4pt, white,shorten >= 4.5pt}]}

\usepackage{booktabs}
\usepackage{multirow}

\pgfplotstableset{
every head row/.style={
  before row={
  \toprule
  },
  after row=\midrule,
},
every last row/.style={
  after row=\bottomrule
},
}

\usepackage{cleveref}

\newcommand{\floatspec}{}
\edef\efigure{\noexpand\begin{figure}[\floatspec]}
\edef\etable{\noexpand\begin{table}[\floatspec]}

\usepackage{float}

\DeclareCiteCommand{\footurlcite}[\mkbibfootnote]
  {\usebibmacro{prenote}}
  {\printfield{url}}
  {\usebibmacro{postnote}}

%%% Local Variables:
%%% mode: latex
%%% TeX-master: "paper"
%%% End:

\begin{document}
\date{16. September 2016}
\pagenumbering{Roman} % Seitennummerierung auf römische Zahlen setzen
\begin{titlepage}
	% Nach einer Vorlage von http://www.LaTeXTemplates.com
	\newcommand{\HRule}{\rule{\linewidth}{0.5mm}} % Defines a new command for the horizontal lines, change thickness here

	\center % Center everything on the page
 

	\textsc{\LARGE Westfälische Wilhelms-Universität Münster}\\[1.5cm] % Name of your university/college
	\textsc{\Large Praktikum zur Mustererkennung}\\[0.5cm] % Major heading such as course name


	\HRule \\[0.4cm]
	
	{\onehalfspacing\huge\sffamily\bfseries Emotionserkennung \\[0.4cm]
    \large Klassifikation von Action Units anhand von Landmarks \singlespacing} % Title of your document
	\vspace{0.4cm}
	% \HRule \\[1.5cm] 
	\HRule \\[3cm] 
	
	\begin{minipage}[t]{0.3\textwidth}
	\begin{center} \large
	Robin Nachname\\ % Your name
	\normalsize \url{mail@adresse}\\
	Matrikelnr. 123456
	\end{center}
	\end{minipage}
	~
	\begin{minipage}[t]{0.3\textwidth}
	\begin{center} \large
	Johannes Stricker\\ 
	\normalsize \url{mail@adresse}\\
	Matrikelnr. 383779
	\end{center}
	\end{minipage}
  ~
	\begin{minipage}[t]{0.3\textwidth}
	\begin{center} \large
	Alexander Schlüter\\ 
	\normalsize \url{alx.schlueter@gmail.com}\\
	Matrikelnr. 409649
	\end{center}
  \end{minipage}

  \vfill

	{\large eingereicht am \thedate}\\[3cm] % Date, change the \today to a set date if you want to be precise


	\includegraphics[height=1.3cm,keepaspectratio]{Bilder/fb10logo.pdf}\\[1cm] % Include a department/university logo - this will require the graphicx package
 

	% \vfill % Fill the rest of the page with whitespace
	
\end{titlepage}
\begin{abstract}
\section*{Vorwort}
Hier entsteht ein Vorwort.
\end{abstract}
\tableofcontents
\cleardoubleoddemptypage
\pagenumbering{arabic}
\setcounter{page}{1}

\chapter{Erstes Kapitel} % (fold)
\label{cha:erstes_kapitel}
\[
	\sum_{i=0}^{\infty} a^i = \int y^2 \,\mathrm{d} x 
\]

\section{Demobereich} % (fold)
\subsection{Demonstration von \texttt{todonotes}} 
Man kann mittels \texttt{\textbackslash{}todo} Notizen an den Rand schreiben.\todo{Hier muss noch was hinzugefügt werden \ldots} Dafür muss allerdings in dem \texttt{\textbackslash{}documentclass}-Befehl die Option \texttt{draft} gesetzt sein. Man kann sich auch eine große Box mitten in den Text setzen lassen, die eine noch anzufertigende Zeichnung markiert. Dazu benutzt man \texttt{\textbackslash{}missingfigure$\{$\ldots$\}$}:
\missingfigure{Hier fehlt eine Zeichnung}

\subsection{Demonstration von \texttt{tikzcd}}
Mit der Umgebung \texttt{tikzcd} kann man sehr einfach schöne kommutative Diagramme zeichnen, die man auch noch bis aufs Feinste konfigurieren kann.
\[
	\begin{tikzcd}[sep=large]
		A \rar["f"] \dar[dashed] & B \dar[hook] \\
		C \urar["g"] & D
	\end{tikzcd}
\]
Kleiner Vergleich von \texttt{\textbackslash{}underbrace} und \texttt{\textbackslash{}Underbrace}
\[
	\underbrace{a+b}_{=1} + (x-y) \qquad \text{versus} \qquad \Underbrace{a+b}{=1} + (x-y)
\]

\subsection{Literaturangaben und Zitate} % (fold)
Wichtige Info vorweg: Das Aussehen der Referenzen hängt sehr stark von dem verwendeten Stil ab! Der einfachste Befehl zum Zitieren ist 
\texttt{\textbackslash{}cite}, dessen Output in etwas das Folgende ist: \cite{CGL}. Man kann noch ein optionales Argument angeben und da zum Beispiel das Kapitel reinschreiben: \cite[Kapitel $\pi$]{CGL}.
\begin{itemize}
	\item \textquote[\cite{CGL}]{kurzes Zitat von Gauß} erzeugt mit \texttt{\textbackslash{}textcite$[\backslash\texttt{cite}\{\ldots\}]\{\ldots\}$}
	\item Für längere Zitat benutzt man besser den Befehl \texttt{\textbackslash{}blockcite$[\backslash\texttt{cite}\{\ldots\}]\{\ldots\}$}, der je nach 
	Länge des Arguments den zitierten Text vom Fließtext absetzt: \blockquote[\cite{CGL}]{Hier kommt jetzt ein ganz langer Text hin, von dem ich hoffe, 
	dass ihn keiner liest, denn er enthält nun mal so gerade gar keine sinnvolle Information. Hier kommt jetzt ein ganz langer Text hin, von dem ich hoffe, 
	dass ihn keiner liest, denn er enthält nun mal so gerade gar keine sinnvolle Information. Hier kommt jetzt ein ganz langer Text hin, von dem ich hoffe, 
	dass ihn keiner liest, denn er enthält nun mal so gerade gar keine sinnvolle Information.}
	Hier geht der Fließtext weiter.
	\item Möchte man einfach nur ein Wort mit Anführungszeichen versehen benutzt man \texttt{\textbackslash{}enquote$\{\ldots \}$}. Zu Beispiel so:
	Die sogenannte \enquote{Lie-Algebra}. Man sollte Anführungszeichen \emph{nie} per Hand setzen!
\end{itemize}

\subsection{Alles zu Theoremen, Definition und so weiter} % (fold)
Eine Definition kann man ganz leicht über \texttt{\textbackslash{}begin$\{$definition$\}$ \ldots \textbackslash{}end$\{$definition$\}$} erstellen:
\begin{definition}
Ein topologischer Raum $X$ heißt \emph{zusammenhängend}, falls er sich nicht als disjunkte Vereinigung von zwei nichtleeren, offenen Mengen schreiben lässt.
\end{definition}
Die Nummerierung erfolgt in der momentanen Einstellung innerhalb des \texttt{section}-Zählers. Genauso sind auch \texttt{Satz,Korollar,Lemma,Proposition} definiert. 
Selbstverständlich lassen sich noch weitere solche Umgebungen definieren. Für Beweise gibt es die Umgebung \texttt{beweis}, die das Zeichen am Ende eines Beweises selbstständig setzt. Dies funktioniert allerdings nur im Fließtext zuverlässig. Unter Umständen muss man mit \texttt{\textbackslash{}qedhere} nachhelfen.
\begin{beweis}
Wir beweisen die Behauptung per Induktion, die keinen Sinn ergibt:
\[
	a^2+b^2=c^2 \qedhere
\]
\end{beweis}
% subsection alles_zu_theoremen_definition_und_so_weiter (end)

\subsection{Verschiede Mathebefehle} % (fold)
Hier werden kurz einige wichtige Mathebefehle vorgestellt:
\begin{description}
	\item[Mengen] Sinnigerweise heißt der Befehl \texttt{\textbackslash{}set}. Ein einfaches Beispiel:
	\[
		\set{v_1,\ldots ,v_n}
	\]
	In der Stern-Variante wird die Größe der Klammern automatisch angepasst, man kann die Größe aber auch manuell mit den bekannten Befehlen \texttt{\textbackslash{}big} 
	usw. setzen:
	\[
		\set*{ \binom{n}{1}, \ldots , \binom{n}{k} } \qquad \set[\Big]{x_1, \ldots ,x_n}
	\]
	Ein \texttt{\textbackslash{}given} im Argument erzeugt die folgende ständig benötigte Notation:
	\[
		\set{x \in \mathbb{R}^2 \given x_1^2 + x_2^2=1} \qquad \set*{ x \in \mathbb{R}^n \given \sqrt{\sum_{i=1}^{n} x_i^2}=1}
	\]
	Ähnliche Konstrukte lassen sich natürlich auch für ähnliche Notation zum Beispiel in der Wahrscheinlichkeitstheorie realisieren.
	\item[Klammern] Oft braucht man auch einfach nur Klammern, die sich der Größe des Inhalts anpassen. In diesem Dokument ist dafür \texttt{\textbackslash{}enbrace} und 
	\command{abs} definiert. Die Syntax ist wie im ersten Fall nur ohne die \command{given} Option
	\[
		\enbrace*{V_1, \ldots ,V_m , \bigoplus_{i=1}^\infty V_i}  \qquad \abs*{x+y} \le \abs{x} + \abs{y} \qquad \abs*{\sum_{i=1}^{\infty} a_i}   
	\]
	\item[Skalarprodukt] Mittels  \command{skal} (zwei Argumente). Hausaufgabe: Herausfinden, wie man auf die Komma-Notation wechselt:
	\[
		\skal{v}{v_1 + \ldots + v_n} = \skal*{\sum v_i}{v}
	\]
	\item[Norm] Von der Syntax her genauso funktioniert \command{norm}
	\[
		\norm*{\frac{x}{\norm{x} } } =1
	\]
\end{description}
% subsection verschiede_mathebefehle (end)

\subsection{Differentialrechnung} % (fold)
Meiner Meinung nach gehört das kleine \enquote{d} in Ableitungen oder Integralen aufrecht: Dazu gibt es den Befehl \command{mathd}:
\[
	\int\! x^2 \, \mathd x
\]
Für (partielle) Ableitungen gibt es dann noch die Befehle \command{diff} und \command{diffd} (jeweils zwei Argumente, die man auch leer lassen darf)
\[
	\diffd{}{t} \cos (c \cdot t) \qquad \diff{f}{x} \qquad \diff{^2 f}{x \partial y}
\]
% subsection differentialrechnung (end)

\subsection{Eigene Operatoren} % (fold)
Wer zum Beispiel für die Sinusfunktion \texttt{sin (x)} statt \texttt{\textbackslash{}sin (x)} schreibt, der begeht eine typografische Todsünde:
\[
	sin (x) \qquad \text{ versus } \qquad \sin (x)
\]
Benutzt man den Befehl, kann man auch die Klammer weglassen, ohne das Spacing zu stören: $\sin x$.
Viele solcher Befehle sind schon vordefiniert. Wenn man selbst neue braucht, geht dies sehr einfach mittels \command{DeclareMathOperator} (siehe Quelltext). Hier wurde ein
Befehl für das Signum definiert
\[
	\sgn \sigma = \pm 1
\]
% subsection eigene_operatoren (end)

\appendix % hier beginnt der Anhang
\printbibliography
\end{document}
