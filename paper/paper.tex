%!TEX TS-program = pdflatex
%!BIB program = biber
% -- Author: Jannes Bantje, j.bantje@wwu.de
%!TEX root = bachelor.tex
% -- Author: Jannes Bantje, j.bantje@wwu.de
\documentclass[a4paper,ngerman,index=totoc,toc=bibliography,fontsize=10,DIV=13,headinclude,twoside,BCOR=12mm,cleardoublepage=empty,draft]{scrreprt}


%-- Basics für graphische Sachen
\usepackage[usenames,x11names]{xcolor} % Die Optionen definieren zusätzliche Farben (siehe Dokumentation)
\usepackage[final]{graphicx}

%-- typografische Verbesserungen, Codierungskram, Schriftwahl und erste Mathepakete
\usepackage[utf8]{inputenc}
\usepackage[lining,semibold]{libertine}
\usepackage[T1]{fontenc}
\usepackage{textcomp} % verhindert ein paar Fehler bei den Fonts
\usepackage[varl]{zi4}
\usepackage{mathtools,amssymb,amsthm} % Verbesserung von amsmath (die amsmath selbst lädt)
\usepackage[libertine,cmintegrals,bigdelims,varbb]{newtxmath}
\usepackage[ngerman]{babel}
\usepackage[babel=true, tracking=true,final]{microtype}

% \useosf % aktiviert sog. "old style figures", also werden Zahlen – im Text – teilweise unterhalb der Grundlinie angezeigt. Muss man mögen...


%-- Alternative Konfiguration mit XeLaTeX, die ein großes "ß" anzeigen kann!
%-- ACHTUNG: wenn diese benutzt werden soll, dan MUSS man in der ersten Zeile von bachelor.tex "pdflatex" durch "xelatex" ersetzen und natürlich den obigen Block auskommentieren und diesen wiederrum aktivieren!!!

% \usepackage{lmodern}
% \usepackage{mathtools,amssymb,amsthm} % Verbesserung von amsmath (die amsmath selbst lädt)
% \usepackage[libertine,cmintegrals,bigdelims,varbb]{newtxmath}
% \usepackage[no-math]{fontspec}
% \usepackage{polyglossia} % moderner babel-ersatz
% \setmainlanguage[spelling=new,babelshorthands=true]{german}
% \shorthandoff{"}
% \defaultfontfeatures{Mapping=tex-text, Ligatures={Required,Common,Contextual}}
% \setmainfont{LinLibertine}[Extension=.otf,UprightFont=*_R,BoldFont=*_RZ,ItalicFont=*_RI,BoldItalicFont=*_RZI,ItalicFeatures={Ligatures=Historical}]
% \setsansfont{LinBiolinum}[Scale=MatchUppercase, Extension=.otf, UprightFont=*_R, BoldFont=*_RB, ItalicFont=*_RI,BoldItalicFont=*_RBO]
% \setmonofont{Inconsolatazi4}[Scale=MatchUppercase,Extension=.otf,UprightFont=*-Regular,BoldFont=*-Bold,StylisticSet=1]
% \usepackage[final]{microtype}


%-- Zeilenabstand einstellen
\usepackage{setspace}
% Nun kann man, wenn gewünscht den Zeilenabstand zum Beispiel auf 1,5 setzen mit \onehalfspacing

\newcommand{\command}[1]{\texttt{\textbackslash{}#1}}



%-- Mathematikpakete und Einstellungen
\mathtoolsset{centercolon} % sorgt dafür dass := und =: besser aussehen
\usepackage{mathdots} % sorgt dafür, dass Punte wie zB \ddots besser aussehen
\newcommand{\Underbrace}[2]{{\underbrace{#1}_{#2}}} % Underbrace als Befehl in LaTeX-Syntax (und ohne Spacing-Probleme mit nachfolgenden Operatoren...)
\renewcommand{\le}{\leqslant} % ich finde Kleinergleich mit schrägen Strich schöner
\renewcommand{\ge}{\geqslant}

%-- charakteristische-Funktion-/Indikatorfunktion-Eins '\ind'
\usepackage{silence}
\WarningFilter{latexfont}{Size substitutions with differences}
\WarningFilter{latexfont}{Font shape `U/bbold/m/n' in size}
\DeclareSymbolFont{bbold}{U}{bbold}{m}{n}
\DeclareSymbolFontAlphabet{\mathbbold}{bbold}
\newcommand{\ind}{\mathbbold{1}}

%-- Ein sehr hübscher Mengen-Befehl
\newcommand\SetSymbol[1][]{\nonscript\:#1\vert\allowbreak\nonscript\:\mathopen{}}
\providecommand\given{} % to make it exist
\DeclarePairedDelimiterX\set[1]\{\}{\renewcommand\given{\SetSymbol[\delimsize]}#1}

%-- Klammern, Skalarprodukt und Norm
\DeclarePairedDelimiter{\enbrace}{(}{)}
\DeclarePairedDelimiter{\abs}{|}{|}
\DeclarePairedDelimiterX\skal[2]{\langle}{\rangle}{#1\,\delimsize\vert\,#2}
\DeclarePairedDelimiter{\norm}{\lVert}{\rVert}

%-- Differentialrechnung
\newcommand{\mathd}{\mathrm{d}\mkern-0.5mu}
\newcommand{\diff}[2]{\frac{{\partial #1}}{{\partial #2}} }
\newcommand{\diffd}[2]{\frac{\mathd #1}{\mathd #2} }

%-- eigene Befehle
\DeclareMathOperator{\sgn}{sgn}


%-- kommutative Diagramme
\usepackage{tikz-cd} %-- meiner Meinung nach das beste Paket für kommutative Diagramme
\tikzset{% um Kompatibilität mit Babel herzustellen und die angenehme "<label>"-Syntax zu nutzen
  every picture/.append style={
    execute at begin picture={\shorthandoff{"}},
    execute at end picture={\shorthandon{"}}
  }
}
\usetikzlibrary{quotes,babel}

%-- Für Literaturangaben, hier wird NICHT das total veraltete bibtex benutzt!
\usepackage[%
	backend=biber,
	sortlocale=auto,
	natbib,
	hyperref,
	% backref,
	style=alphabetic % eine unvollständige Auswahl von Styles: ieee, numeric, apa
	]%
{biblatex}
\addbibresource{quellen.bib} % Literaturdatei einlesen

% -- Konfiguration von Hyperref (sorgt für anklickbare Links und ein PDF-Inhaltsverzeichnis)
\usepackage[hidelinks, pdfpagelabels, bookmarksopen=true, bookmarksnumbered=true, linkcolor=black, urlcolor=SkyBlue2, plainpages=false,pagebackref, citecolor=black, hypertexnames=true, pdfborderstyle={/S/U}, linkbordercolor=SkyBlue2, colorlinks=false, backref=false]{hyperref}
\hypersetup{final}

%-- Für Aufzählungen und andere Listen, Anführungszeichen und Zitate
\usepackage[shortlabels]{enumitem} % durch die Option ist die gleiche Syntax wie zB mit dem Paket paralist möglich
\setlist[enumerate,description]{font=\sffamily\bfseries} % sorgt dafür, dass die Labels bei enumerate und description fett sind
\usepackage[german=quotes]{csquotes}

%-- Für hilfreiche Anmerkungen am Seitenrand
\usepackage[obeyDraft,textsize=small]{todonotes}

%-- Kopf- und Fußzeilen bearbeiten
\usepackage{scrpage2}
\pagestyle{scrheadings}
\clearscrheadfoot % Standardkonfiguration löschen
\setheadsepline{1pt} % Linie für die Kopfzeile
\automark[section]{chapter} % definiert, welcher Text in den Kolumnentiteln erscheinen soll
\rohead{\rightmark} % section erscheint rechts oben
\lehead{\scshape\leftmark} % chapter erscheint links oben in ist in small caps gesetzt
\ofoot[\pagemark]{\pagemark} % Seitenzahlen immer außen, hier wir auch der plain Stil bearbeitet!
% \ifoot[Titel der Bachelorarbeit]{Titel der Bachelorarbeit}
\renewcommand*{\pnumfont}{\LARGE\sffamily} % Seitenzahlen in groß und serifenlos
\renewcommand*{\footfont}{\large\sffamily\color{gray}}
% \renewcommand*{\headfont}{\normalfont}



%-- Theorem-Pakete und Konfiguration
\usepackage{thmtools}

\usepackage{bookmark}
% Theoreme als PDF-Lesezeichen
\bookmarksetup{open,numbered}
\makeatletter
\newcommand*{\theorembookmark}{%
  \bookmark[
    dest=\@currentHref,
    rellevel=1,
    keeplevel,
  ]{%
    \thmt@thmname\space\csname the\thmt@envname\endcsname
    \ifx\thmt@shortoptarg\@empty
    \else
      \space(\thmt@shortoptarg)%
    \fi
  }%
}
\makeatother



\declaretheoremstyle[%
	headfont=\sffamily\bfseries,
	notefont=\normalfont\sffamily,
	bodyfont=\normalfont,
	headformat=\NUMBER\ \NAME\NOTE,
	headpunct={},
	postheadspace=1ex,
	postheadhook=\theorembookmark,
	spaceabove=15pt,spacebelow=10pt,]%
{mainstyle}
\declaretheoremstyle[%
	headfont=\bfseries\scshape,
	bodyfont=\normalfont,
	headpunct=:,
	postheadspace=1ex,
	spacebelow=12pt,spaceabove=2pt,
	qed=\qedsymbol]%
{beweise}

\declaretheorem[name=Definition,parent=section,style=mainstyle]{definition}
\declaretheorem[name=Satz,sharenumber=definition,style=mainstyle]{satz}
\declaretheorem[name=Korollar,sharenumber=definition,style=mainstyle]{korollar}
\declaretheorem[name=Lemma,sharenumber=definition,style=mainstyle]{lemma}
\declaretheorem[name=Proposition,sharenumber=definition,style=mainstyle]{proposition}

\declaretheorem[name=Beweis,numbered=no,style=beweise]{beweis}

\usepackage{titling}
\usepackage{pgfplots}
\usepackage{pgfplotstable}

\pgfplotsset{compat=1.13}
\usetikzlibrary{arrows,decorations.markings,positioning,calc,}

\pgfplotsset{
% Style to select only points from #1 to #2 (inclusive)
  select coords between index/.style 2 args={
    x filter/.code={
        \ifnum\coordindex<#1\def\pgfmathresult{}\fi
        \ifnum\coordindex>#2\def\pgfmathresult{}\fi
    }
},
  discard if not/.style 2 args={
      x filter/.code={
          \edef\tempa{\thisrow{#1}}
          \edef\tempb{#2}
          \ifx\tempa\tempb
          \else
              \def\pgfmathresult{inf}
          \fi
      }
    },
    filter discard warning=false
}

\newcount\colveccount
\newcommand*\colvec[1]{
        \global\colveccount#1
        \begin{pmatrix}
        \colvecnext
}
\def\colvecnext#1{
        #1
        \global\advance\colveccount-1
        \ifnum\colveccount>0
                \\
                \expandafter\colvecnext
        \else
                \end{pmatrix}
        \fi
}
\newcommand{\myarrowopts}{[thick, decoration={markings,mark=at position
   1 with {\arrow[semithick]{open triangle 60}}},
   double distance=1.4pt, shorten >= 5.5pt,
   preaction = {decorate},
   postaction = {draw,line width=1.4pt, white,shorten >= 4.5pt}]}

\usepackage{booktabs}
\usepackage{multirow}

\pgfplotstableset{
every head row/.style={
  before row={
  \toprule
  },
  after row=\midrule,
},
every last row/.style={
  after row=\bottomrule
},
}

\usepackage{cleveref}

\newcommand{\floatspec}{}
\edef\efigure{\noexpand\begin{figure}[\floatspec]}
\edef\etable{\noexpand\begin{table}[\floatspec]}

\usepackage{float}

\DeclareCiteCommand{\footurlcite}[\mkbibfootnote]
  {\usebibmacro{prenote}}
  {\printfield{url}}
  {\usebibmacro{postnote}}

%%% Local Variables:
%%% mode: latex
%%% TeX-master: "paper"
%%% End:

\begin{document}
\date{16. September 2016}
\pagenumbering{Roman} % Seitennummerierung auf römische Zahlen setzen

\begin{titlepage}
	% Nach einer Vorlage von http://www.LaTeXTemplates.com
	\newcommand{\HRule}{\rule{\linewidth}{0.5mm}} % Defines a new command for the horizontal lines, change thickness here

	\center % Center everything on the page
 

	\textsc{\LARGE Westfälische Wilhelms-Universität Münster}\\[1.5cm] % Name of your university/college
	\textsc{\Large Praktikum zur Mustererkennung}\\[0.5cm] % Major heading such as course name


	\HRule \\[0.4cm]
	
	{\onehalfspacing\huge\sffamily\bfseries Emotionserkennung \\[0.4cm]
    \large Klassifikation von Action Units anhand von Landmarks \singlespacing} % Title of your document
	\vspace{0.4cm}
	% \HRule \\[1.5cm] 
	\HRule \\[3cm] 
	
	\begin{minipage}[t]{0.3\textwidth}
	\begin{center} \large
	Robin Rexeisen\\ % Your name
	\normalsize \url{r_rexe01@wwu.de}\\
	Matrikelnr. 123456
	\end{center}
	\end{minipage}
	\quad
	\begin{minipage}[t]{0.3\textwidth}
	\begin{center} \large
	Johannes Stricker\\ 
	\normalsize \url{johannesstricker@gmx.net}\\
	Matrikelnr. 383779
	\end{center}
	\end{minipage}
  \quad
	\begin{minipage}[t]{0.3\textwidth}
	\begin{center} \large
	Alexander Schlüter\\ 
	\normalsize \url{alx.schlueter@gmail.com}\\
	Matrikelnr. 409649
	\end{center}
  \end{minipage}\\[2cm]
  \large
  \emph{Betreuer:}\\
	Sören Klemm\\ 
	\normalsize \url{soeren.klemm@wwu.de}\\

  \vfill

	{\large eingereicht am \thedate}\\[1cm] % Date, change the \today to a set date if you want to be precise


	\includegraphics[height=1.3cm,keepaspectratio]{Bilder/fb10logo.pdf}\\[1cm] % Include a department/university logo - this will require the graphicx package
 

	% \vfill % Fill the rest of the page with whitespace
	
\end{titlepage}
%%% Local Variables: 
%%% mode: latex
%%% TeX-master: "paper"
%%% End: 

\begin{abstract}
\section*{Vorwort}
Hier entsteht ein Vorwort.
\end{abstract}
\tableofcontents
\cleardoubleoddemptypage
\pagenumbering{arabic}
\setcounter{page}{1}

\chapter{Erstes Kapitel}
\pgfplotstableread[x expr=\thisrowno{0}*cos(6)+\thisrowno{1}*sin(6),y expr=\thisrowno{0}*-sin(6)+\thisrowno{1}*cos(6)]{data/face_lipcorner_low.dat}{\facerot}
\pgfplotstablecreatecol[
create col/expr={\thisrow{0}*cos(6)+\thisrow{1}*sin(6)}
]{xrot}{\facerot}
\pgfplotstablecreatecol[
create col/expr={\thisrow{0}*-sin(6)+\thisrow{1}*cos(6)}
]{yrot}{\facerot}
\begin{figure}
  \centering
\begin{tikzpicture}
  \begin{axis}[
    name=plot1,
    width=6cm,
    height=6cm,
    ticks=none,
    y dir=reverse
    ]
    \addplot [
    mark=o,
    only marks,
    mark size=1pt
    ] table [x=xrot, y=yrot] \facerot;
    \addplot [
    red,
    x filter/.expression={\coordindex==36 || \coordindex==45 ? x : nan}
    ] table [x=xrot, y=yrot] \facerot;
    \addplot [
    blue,
    dashed,
    x filter/.expression={\coordindex==16 || \coordindex==0 ? x : nan}
    ] table [x=xrot, y=yrot] \facerot;
 \end{axis}
  \begin{axis}[
    name=plot2,
    at={($(plot1.right of east)+(3cm,0)$)}, anchor=left of west,
    width=6cm,
    height=6cm,
    ticks=none,
    y dir=reverse
    ]
    \addplot [
    mark=o,
    only marks,
    mark size=1pt
    ] table \facerot;
    \addplot [
    red,
    x filter/.expression={\coordindex==36 || \coordindex==45 ? x : nan}
    ] table \facerot;
    \addplot [
    blue,
    dashed,
    x filter/.expression={\coordindex==16 || \coordindex==0 ? x : nan}
    ] table \facerot;
 \end{axis}
\draw  \myarrowopts ($(plot1.east)+(0.5cm,0)$) -- ($(plot2.west)+(-0.5cm,0)$) node[above,midway] {normalize} node[below,midway] {rotation};
\end{tikzpicture}
\caption{Normalisierung der Rotation anhand ausgesuchter Linien}
\end{figure}
\begin{figure}
  \centering
\begin{tikzpicture}
  \begin{scope}[local bounding box=scope1]
  \begin{axis}[
    width=7cm,
    height=4cm,
    ticks=none,
    y dir=reverse,
    axis lines=none
    ]
    \addplot [
    draw=blue,
    mark=*,
    only marks,
    mark size=1pt,
    nodes near coords={
      \pgfmathparse{int(\coordindex+48)}
      \pgfmathresult
    },
    every node near coord/.style={
        font=\scriptsize,
        anchor=south west,
        xshift=-2pt,
        yshift=-3pt
      },
      select coords between index={48}{54},
    ] table {data/face_lipcorner_low.dat};
    \addplot [
    fill=blue,
    mark=o,
    only marks,
    mark size=1pt,
    select coords between index={55}{65},
    ] table {data/face_lipcorner_low.dat};
  \end{axis}
\end{scope}
  \begin{scope}[local bounding box=scope2, shift={($(scope1.east)+(3cm,-0.6cm)$)}]
    
  \begin{axis}[
    width=0.4\textwidth,
    height=3.5cm,
    axis lines=left,
    ticks=none,
    xmin=-0.55,
    ymin=-4e-2,
    legend entries={$ax^4+bx^3+cx²+dx+e$},
    legend style={
      at={(0,-0.2)},
      anchor=north west
    }
    ]
    \addplot [
    forget plot,
    draw=blue,
    mark=*,
    only marks,
    mark size=1pt,
    nodes near coords={
      \pgfmathparse{int(\coordindex+48)}
      \pgfmathresult
    },
    every node near coord/.style={
        font=\scriptsize,
        anchor=south west,
        xshift=-2pt,
        yshift=-3pt
    }
    ] table {data/lipcorner_low_norm.dat};
    \addplot [red,
    domain=-0.5:0.5
    ] {1.893688218710873339e-02 +
      x * (-3.121906790233518192e-02 +
      x * (-5.189325604221312060e-02 +
      x * (1.475542295096524681e-01 +
      x * (-6.156357895166050254e-01))))};

  \end{axis}
  \end{scope}
  \begin{scope}[local bounding box=scope3,shift={($(scope2.east)+(3cm,0)$)}]
% \node {$\colvec{5}{-0.616}{0.148}{-0.052}{-0.031}{0.019}$};
\node  {$\colvec{5}{a}{b}{c}{d}{e}$};
  \end{scope}
\draw  \myarrowopts ($(scope1.east)+(0.5cm,0)$) -- ($(scope2.west)+(-0.5cm,0)$) node[above,midway] {normalize + fit};
\draw \myarrowopts ($(scope2.east)+(0.5cm,0)$) -- ($(scope3.west)+(-0.5cm,0)$) node[above,midway] {extract coeffs.};

\end{tikzpicture}
\caption{InterpolationFeatureExtraction}
\end{figure}

\begin{figure}
  \centering
  \begin{tikzpicture}
    \begin{axis}[
      title=Lips Part,
      only marks,
      mark size=1.5pt,
      xlabel=$1-\text{Precision}$,
      ylabel=Recall
      ]
     \addplot table [x index=1, y index=2, x expr={1-\thisrow{Precision}}] {data/validation/Lips_Part.dat};
    \end{axis}
  \end{tikzpicture}
  \caption{Ergebnisse der Klassifikatoren und Parameter für Lips Part}
\end{figure}



\begin{figure}
  \centering
  \pgfplotstabletypeset[
  precision=3,
  columns={F1Validation, F1Test, PrecisionTest,
    RecallTest}, columns/F1Validation/.style={column name={F1 score
      Validation}}, columns/F1Test/.style={column name={F1 score Test}},
  columns/PrecisionTest/.style={column name={Precision Test}},
  columns/RecallTest/.style={column name={Recall Test}}
  ]{data/top5/Lips_Part.dat}
  
  \caption{F1 scores und Testergebnisse der Top 5 Klassifikatoren für Lips Part}
\end{figure}



\appendix % hier beginnt der Anhang
\printbibliography
\end{document}
